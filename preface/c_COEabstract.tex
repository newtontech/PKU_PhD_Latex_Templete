% !Mode:: "TeX:UTF-8"
% 文字编码:UTF-8
%%%%%%%%%%%%%%%%%%%%%%%%%%%%%%%%%%%%%%%%%%%%%%%%%%%%%%%%%%%%%%%%%%%%%%%%%
%
%   LaTeX File for Doctor (Master) Thesis of Peking University
%   LaTeX + CJK     北京大学博士(硕士)论文模板
%   Based on Wang Lei's Template for THU
%   Version: 1.00
%   Last Update: 2005-05-25
%
%%%%%%%%%%%%%%%%%%%%%%%%%%%%%%%%%%%%%%%%%%%%%%%%%%%%%%%%%%%%%%%%%%%%%%%%%
%   Copyright 2004-2005  by  Ying Pan       (yeying_pan@yahoo.com.cn)
%%%%%%%%%%%%%%%%%%%%%%%%%%%%%%%%%%%%%%%%%%%%%%%%%%%%%%%%%%%%%%%%%%%%%%%%%
%%%%%%%%%%%%%%%%%%%%%%%%%%%%%%%%%%%%%%%%%%%%%%%%%%%%%%%%%%%%%%%%%%%%%%%%%
%
%   LaTeX File for Doctor (Master) Thesis of Tsinghua University
%   LaTeX + CJK     清华大学博士(硕士)论文模板
%   Based on Wang Tianshu's Template for XJTU
%   Version: 1.00
%   Last Update: 2003-09-12
%
%%%%%%%%%%%%%%%%%%%%%%%%%%%%%%%%%%%%%%%%%%%%%%%%%%%%%%%%%%%%%%%%%%%%%%%%%
%   Copyright 2002-2003  by  Lei Wang (BaconChina)       (bcpub@sina.com)
%%%%%%%%%%%%%%%%%%%%%%%%%%%%%%%%%%%%%%%%%%%%%%%%%%%%%%%%%%%%%%%%%%%%%%%%%

\newitemsep
\renewcommand{\labelenumi}{(\arabic{enumi})}
\cabstract{
\thispagestyle{plain}
某某问题是……

本文\footnotecircle{本研究得到某某基金(编号:XXX)资助。}采用了……

研究表明……

使用说明:

1. 主文件为COEmain.tex,里面可详细看到对应封面,致谢,出版情况与个人简历对应的文件。

2. 默认指定使用图片在figures文件夹内,如现在使用的北大logo(logo.eps)、版权声明(copyright2.pdf)以及北京大学学位论文原创性声明和使用授权说明(Authority2.pdf)。

3. 参考文献使用基于BIB文件管理,建议对应的BIB代码使用学术谷歌统一标定。

4. 编译软件大家可以选择使用TEXLIVE,需要保证完全安装。编译过程是一次xelatex命令后,再文献编译bibtex一次,再xelatex两次,即可查看pdf结果。

5. 当论文中包含大量eps格式的图片时,xelatex编译时间较长;将所有eps图片的格式转为pdf可以显著缩短编译时间。

6. 打印生成的pdf时,为了保证与学院Word模板相同的效果,请选择“实际大小”或者“自定义比例100\%”。
\smallskip

{\color{red}{英文培养项目学生使用英文写学位论文,中文摘要应不少于6000字。}}

{\color{red}{Students of the English Training Program are required to write their dissertations in English, and their Chinese abstracts should be no less than 6,000 Chinese characters.}}
\smallskip
%\ \quad
% 当关键词与脚注(某某基金资助)在同一页时,需要人工增减“\quad \\”来保证“关键词放摘要页最下方”。另一种方法是采用\vfill,但是在package.tex中使用脚注控宏包footmisc时就不能再采用选项bottom,代价是每页的脚注底部可能不对齐。

% 当中文摘要较长,使得关键词在摘要的第二页时,采用\vfill即可使关键词位于页面底部。(与英文摘要中关键词的处理方法相同)
 
\vspace{7.0cm} 关键词:关键词1,关键词2,关键词3……{\color{red}{关键词需要置底}}
}

