% !Mode:: "TeX:UTF-8"
% 文字编码:UTF-8
\chapter[图表和公式示例]{图表和公式示例\protect\footnotecircle{图标题在图下方,表标题在表上方。图表序号分章设置,如图3.15表示第三章第15幅图。}}

本章给出图表和公式的示例。

图片标题在图片下方。

\begin{figure}[htbp]
  \centering
  \includegraphics[scale=0.35]{figure_A.pdf}
  \caption{单张图片}
  \label{fig:single_figure}
\end{figure}

图表序号分章设置,如图\ref{fig:single_figure}表示第二章第1幅图。

\begin{figure}[htbp]
  \begin{subfigure}[b]{0.5\linewidth}
    \centering
    \includegraphics[scale=0.35]{figure_A.pdf}
    \caption{初始能量$E=1$的能量曲线}
  \end{subfigure}
  \begin{subfigure}[b]{0.5\linewidth}
    \centering
    \includegraphics[scale=0.35]{figure_B.pdf}
    \caption{初始能量$E=100$的能量曲线}
  \end{subfigure}
  \caption{由两个子图组成的图}
  \label{fig:two_figures}
\end{figure}

如果一个图由两个或两个以上分图组成时,各分图分别以(a)、(b)、(c)……作为图序,并须有分图名。

以上是单张图片和由两个子图组成的图片示例。

下面给出表格的示例。

表格的标题在表格上方。

\begin{table}[htbp]
  \typetable
  \centering  
  \def~{\hphantom{0}}
  \caption{表格的例子}
  \label{tab:list}
  \begin{tabular}{ccc}
    \toprule
    类型 & 线性稳定性临界参数 & 能量稳定性临界参数 \\
    \midrule
    平面Poiseuille流 & 5772.22 & 49.60  \\
    平面Couette流 & 稳定 & 20.7 \\
    管流 & 对于轴对称扰动稳定 & 81.49 \\
    Rayleigh--Bernard对流 & 657.5 & 657.5 \\
    \bottomrule
  \end{tabular}
\end{table}

图片和表格的详细说明见第3.7.4节。

行内公式示例:设远场流体温度稳定分层梯度的大小为$N$,则竖直方向高度为$s$的流体温度为$T_{\infty}(s)=T_{\infty}(0)+N s$。设斜板壁面在$s$高度处的温度总比同一高度处的流体温度高出一个固定值,远场流体静止,壁面处无滑移。

行间公式示例:
\begin{equation}
\label{eq:Boussinesq}
  \left\{
  \begin{aligned}
    & \frac{\partial u}{\partial x} + \frac{\partial v}{\partial y}=0, \\
    & \frac{\partial u}{\partial t} + \bm{u \cdot \nabla} u = - \frac{\partial p}{\partial x} + \frac{1}{\mbox{\textit{Re}}} \bm{\nabla}^2 u, \\
    & \frac{\partial v}{\partial t} + \bm{u \cdot \nabla} v = - \frac{\partial p}{\partial y} + \frac{1}{\mbox{\textit{Re}}} \bm{\nabla}^2 v.
  \end{aligned}
\right.
\end{equation}

Latex中数学字体的输入方法:斜体$U$,斜体加粗$\bm{U}$,直立体加粗$\mathbf{U}$,实数集$\mathbb{R}$,希腊字母的加粗$\bm{\Lambda}$。有的常数和函数名需要写成直立体,例如$\mathop{\rm Im}\; \mathrm{e}^{\mathrm{i}\theta} = \sin\theta$。
%可以考虑利用newcommand和def来缩写上面的式子,例如“\newcommand{\me}{\mathrm{e}}”以及“\def\Imag{\mathop{\rm Im}\;}”

\textcolor{red}{图例和表头的结尾都不推荐有句号,如果实在觉得需要,请保持全文统一,都有句号,或者都没有句号。 }

